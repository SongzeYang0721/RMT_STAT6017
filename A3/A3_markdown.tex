% Options for packages loaded elsewhere
\PassOptionsToPackage{unicode}{hyperref}
\PassOptionsToPackage{hyphens}{url}
%
\documentclass[
]{article}
\usepackage{amsmath,amssymb}
\usepackage{iftex}
\ifPDFTeX
  \usepackage[T1]{fontenc}
  \usepackage[utf8]{inputenc}
  \usepackage{textcomp} % provide euro and other symbols
\else % if luatex or xetex
  \usepackage{unicode-math} % this also loads fontspec
  \defaultfontfeatures{Scale=MatchLowercase}
  \defaultfontfeatures[\rmfamily]{Ligatures=TeX,Scale=1}
\fi
\usepackage{lmodern}
\ifPDFTeX\else
  % xetex/luatex font selection
\fi
% Use upquote if available, for straight quotes in verbatim environments
\IfFileExists{upquote.sty}{\usepackage{upquote}}{}
\IfFileExists{microtype.sty}{% use microtype if available
  \usepackage[]{microtype}
  \UseMicrotypeSet[protrusion]{basicmath} % disable protrusion for tt fonts
}{}
\makeatletter
\@ifundefined{KOMAClassName}{% if non-KOMA class
  \IfFileExists{parskip.sty}{%
    \usepackage{parskip}
  }{% else
    \setlength{\parindent}{0pt}
    \setlength{\parskip}{6pt plus 2pt minus 1pt}}
}{% if KOMA class
  \KOMAoptions{parskip=half}}
\makeatother
\usepackage{xcolor}
\usepackage[margin=1in]{geometry}
\usepackage{color}
\usepackage{fancyvrb}
\newcommand{\VerbBar}{|}
\newcommand{\VERB}{\Verb[commandchars=\\\{\}]}
\DefineVerbatimEnvironment{Highlighting}{Verbatim}{commandchars=\\\{\}}
% Add ',fontsize=\small' for more characters per line
\usepackage{framed}
\definecolor{shadecolor}{RGB}{248,248,248}
\newenvironment{Shaded}{\begin{snugshade}}{\end{snugshade}}
\newcommand{\AlertTok}[1]{\textcolor[rgb]{0.94,0.16,0.16}{#1}}
\newcommand{\AnnotationTok}[1]{\textcolor[rgb]{0.56,0.35,0.01}{\textbf{\textit{#1}}}}
\newcommand{\AttributeTok}[1]{\textcolor[rgb]{0.13,0.29,0.53}{#1}}
\newcommand{\BaseNTok}[1]{\textcolor[rgb]{0.00,0.00,0.81}{#1}}
\newcommand{\BuiltInTok}[1]{#1}
\newcommand{\CharTok}[1]{\textcolor[rgb]{0.31,0.60,0.02}{#1}}
\newcommand{\CommentTok}[1]{\textcolor[rgb]{0.56,0.35,0.01}{\textit{#1}}}
\newcommand{\CommentVarTok}[1]{\textcolor[rgb]{0.56,0.35,0.01}{\textbf{\textit{#1}}}}
\newcommand{\ConstantTok}[1]{\textcolor[rgb]{0.56,0.35,0.01}{#1}}
\newcommand{\ControlFlowTok}[1]{\textcolor[rgb]{0.13,0.29,0.53}{\textbf{#1}}}
\newcommand{\DataTypeTok}[1]{\textcolor[rgb]{0.13,0.29,0.53}{#1}}
\newcommand{\DecValTok}[1]{\textcolor[rgb]{0.00,0.00,0.81}{#1}}
\newcommand{\DocumentationTok}[1]{\textcolor[rgb]{0.56,0.35,0.01}{\textbf{\textit{#1}}}}
\newcommand{\ErrorTok}[1]{\textcolor[rgb]{0.64,0.00,0.00}{\textbf{#1}}}
\newcommand{\ExtensionTok}[1]{#1}
\newcommand{\FloatTok}[1]{\textcolor[rgb]{0.00,0.00,0.81}{#1}}
\newcommand{\FunctionTok}[1]{\textcolor[rgb]{0.13,0.29,0.53}{\textbf{#1}}}
\newcommand{\ImportTok}[1]{#1}
\newcommand{\InformationTok}[1]{\textcolor[rgb]{0.56,0.35,0.01}{\textbf{\textit{#1}}}}
\newcommand{\KeywordTok}[1]{\textcolor[rgb]{0.13,0.29,0.53}{\textbf{#1}}}
\newcommand{\NormalTok}[1]{#1}
\newcommand{\OperatorTok}[1]{\textcolor[rgb]{0.81,0.36,0.00}{\textbf{#1}}}
\newcommand{\OtherTok}[1]{\textcolor[rgb]{0.56,0.35,0.01}{#1}}
\newcommand{\PreprocessorTok}[1]{\textcolor[rgb]{0.56,0.35,0.01}{\textit{#1}}}
\newcommand{\RegionMarkerTok}[1]{#1}
\newcommand{\SpecialCharTok}[1]{\textcolor[rgb]{0.81,0.36,0.00}{\textbf{#1}}}
\newcommand{\SpecialStringTok}[1]{\textcolor[rgb]{0.31,0.60,0.02}{#1}}
\newcommand{\StringTok}[1]{\textcolor[rgb]{0.31,0.60,0.02}{#1}}
\newcommand{\VariableTok}[1]{\textcolor[rgb]{0.00,0.00,0.00}{#1}}
\newcommand{\VerbatimStringTok}[1]{\textcolor[rgb]{0.31,0.60,0.02}{#1}}
\newcommand{\WarningTok}[1]{\textcolor[rgb]{0.56,0.35,0.01}{\textbf{\textit{#1}}}}
\usepackage{graphicx}
\makeatletter
\def\maxwidth{\ifdim\Gin@nat@width>\linewidth\linewidth\else\Gin@nat@width\fi}
\def\maxheight{\ifdim\Gin@nat@height>\textheight\textheight\else\Gin@nat@height\fi}
\makeatother
% Scale images if necessary, so that they will not overflow the page
% margins by default, and it is still possible to overwrite the defaults
% using explicit options in \includegraphics[width, height, ...]{}
\setkeys{Gin}{width=\maxwidth,height=\maxheight,keepaspectratio}
% Set default figure placement to htbp
\makeatletter
\def\fps@figure{htbp}
\makeatother
\setlength{\emergencystretch}{3em} % prevent overfull lines
\providecommand{\tightlist}{%
  \setlength{\itemsep}{0pt}\setlength{\parskip}{0pt}}
\setcounter{secnumdepth}{-\maxdimen} % remove section numbering
\ifLuaTeX
  \usepackage{selnolig}  % disable illegal ligatures
\fi
\IfFileExists{bookmark.sty}{\usepackage{bookmark}}{\usepackage{hyperref}}
\IfFileExists{xurl.sty}{\usepackage{xurl}}{} % add URL line breaks if available
\urlstyle{same}
\hypersetup{
  pdftitle={6017\_A3},
  hidelinks,
  pdfcreator={LaTeX via pandoc}}

\title{6017\_A3}
\author{}
\date{\vspace{-2.5em}2023-09-11}

\begin{document}
\maketitle

\subsection{Question 1}\label{question-1}

Suppose we had two independent \(p\)-dimensional vector samples
\(X := {x_1,...,𝕩_{n_1}}\) and \(Y := {y_1,...,y_{n_2}}\) where
\(p \leq n_2\). We assume that each sample comes from a (possibly
different) population distribution with \(i.i.d.\) components and finite
second moment.

\begin{enumerate}
\def\labelenumi{(\alph{enumi})}
\tightlist
\item
  How is the Fisher LSD related to the two vector samples \(X\) and
  \(Y\)? What is the relationship between the two parameters \((s,t)\)
  of \(F_{s,t}\) and the three values \((p,n_1,n_2)\) describing the
  dimensionality and sizes of \(X\) and \(Y\)?
\end{enumerate}

Denote the \(S_1\) as the sample covariance matrix for the \(X\) and the
\(S_2\) as the sample covariance matrix for the \(Y\). The random Fisher
matrices take the form of \(V_n = S_1 S_2^{-1}, n = (n_1,n_2)\). When
the \(p/n_1 \rightarrow y_1\) and \(p/n_2 \rightarrow y_2\), the
empirical spectral distribution (ESD) of \(F_n^{V_n}\) of \(V_n\)
converges to the a limiting spectral distribution (LSD)
\(F_{y_1, y_2}\), that is, \(s = y_1, t = y_2\).

\begin{enumerate}
\def\labelenumi{(\alph{enumi})}
\setcounter{enumi}{1}
\tightlist
\item
  Now that you explained the relationship between \(X, Y\) , the values
  \((p, n_1, n_2)\) and the parameters \((s,t)\) of \(F_{s,t}\) in part
  (a), what would you expect the empirical density of eigenvalues be for
  the following three choices of triplets \((p, n1, n2)\):
  \((50, 100, 100), (75, 100, 200), (25, 100, 200)\).
\end{enumerate}

Firstly, we can write a function about LSD of the Fisher matrix.

\begin{Shaded}
\begin{Highlighting}[]
\NormalTok{pdf\_Fisher }\OtherTok{\textless{}{-}} \ControlFlowTok{function}\NormalTok{(x, }\AttributeTok{p =}\NormalTok{ p, }\AttributeTok{n1 =}\NormalTok{ n1, }\AttributeTok{n2 =}\NormalTok{ n2)\{}
\NormalTok{  s }\OtherTok{=}\NormalTok{ p}\SpecialCharTok{/}\NormalTok{n1}
\NormalTok{  t }\OtherTok{=}\NormalTok{ p}\SpecialCharTok{/}\NormalTok{n2}
\NormalTok{  h }\OtherTok{=}\NormalTok{ (s }\SpecialCharTok{+}\NormalTok{ t }\SpecialCharTok{{-}}\NormalTok{ s}\SpecialCharTok{*}\NormalTok{t)}\SpecialCharTok{\^{}}\NormalTok{\{}\DecValTok{1}\SpecialCharTok{/}\DecValTok{2}\NormalTok{\}}
\NormalTok{  a }\OtherTok{=}\NormalTok{ (}\DecValTok{1} \SpecialCharTok{{-}}\NormalTok{ h)}\SpecialCharTok{\^{}}\DecValTok{2} \SpecialCharTok{/}\NormalTok{ (}\DecValTok{1} \SpecialCharTok{{-}}\NormalTok{ t)}\SpecialCharTok{\^{}}\DecValTok{2}
\NormalTok{  b }\OtherTok{=}\NormalTok{ (}\DecValTok{1} \SpecialCharTok{+}\NormalTok{ h)}\SpecialCharTok{\^{}}\DecValTok{2} \SpecialCharTok{/}\NormalTok{ (}\DecValTok{1} \SpecialCharTok{{-}}\NormalTok{ t)}\SpecialCharTok{\^{}}\DecValTok{2}
  \FunctionTok{ifelse}\NormalTok{(x }\SpecialCharTok{\textless{}=}\NormalTok{ a }\SpecialCharTok{|}\NormalTok{ x }\SpecialCharTok{\textgreater{}=}\NormalTok{b, }\DecValTok{0}\NormalTok{, }\FunctionTok{suppressWarnings}\NormalTok{((}\DecValTok{1}\SpecialCharTok{{-}}\NormalTok{t)}\SpecialCharTok{/}\NormalTok{ (}\DecValTok{2} \SpecialCharTok{*}\NormalTok{ pi }\SpecialCharTok{*}\NormalTok{ x }\SpecialCharTok{*}\NormalTok{ (s }\SpecialCharTok{+}\NormalTok{ t}\SpecialCharTok{*}\NormalTok{x)) }\SpecialCharTok{*}\FunctionTok{sqrt}\NormalTok{((b }\SpecialCharTok{{-}}\NormalTok{ x) }\SpecialCharTok{*}\NormalTok{ (x }\SpecialCharTok{{-}}\NormalTok{a)), }\StringTok{"x"}\NormalTok{))}
\NormalTok{\}}
\end{Highlighting}
\end{Shaded}

Then, we can write a function to create a equally spaced points inside
the support of of the Fisher LSD.

\begin{Shaded}
\begin{Highlighting}[]
\NormalTok{fisher\_support }\OtherTok{\textless{}{-}} \ControlFlowTok{function}\NormalTok{(}\AttributeTok{n\_points =} \DecValTok{200}\NormalTok{, }\AttributeTok{p =}\NormalTok{ p, }\AttributeTok{n1 =}\NormalTok{ n1, }\AttributeTok{n2 =}\NormalTok{ n2)\{}
\NormalTok{  s }\OtherTok{=}\NormalTok{ p}\SpecialCharTok{/}\NormalTok{n1}
\NormalTok{  t }\OtherTok{=}\NormalTok{ p}\SpecialCharTok{/}\NormalTok{n2}
\NormalTok{  h }\OtherTok{=}\NormalTok{ (s }\SpecialCharTok{+}\NormalTok{ t }\SpecialCharTok{{-}}\NormalTok{ s}\SpecialCharTok{*}\NormalTok{t)}\SpecialCharTok{\^{}}\NormalTok{\{}\DecValTok{1}\SpecialCharTok{/}\DecValTok{2}\NormalTok{\}}
\NormalTok{  a }\OtherTok{=}\NormalTok{ (}\DecValTok{1} \SpecialCharTok{{-}}\NormalTok{ h)}\SpecialCharTok{\^{}}\DecValTok{2} \SpecialCharTok{/}\NormalTok{ (}\DecValTok{1} \SpecialCharTok{{-}}\NormalTok{ t)}\SpecialCharTok{\^{}}\DecValTok{2}
\NormalTok{  b }\OtherTok{=}\NormalTok{ (}\DecValTok{1} \SpecialCharTok{+}\NormalTok{ h)}\SpecialCharTok{\^{}}\DecValTok{2} \SpecialCharTok{/}\NormalTok{ (}\DecValTok{1} \SpecialCharTok{{-}}\NormalTok{ t)}\SpecialCharTok{\^{}}\DecValTok{2}
  \FunctionTok{seq}\NormalTok{(a, b, }\AttributeTok{length.out =}\NormalTok{ n\_points)}
\NormalTok{\}}
\end{Highlighting}
\end{Shaded}

Let's plot the three cases together.

\begin{Shaded}
\begin{Highlighting}[]
\NormalTok{p }\OtherTok{=} \DecValTok{50}
\NormalTok{n1 }\OtherTok{=} \DecValTok{100}
\NormalTok{n2 }\OtherTok{=} \DecValTok{100}
\NormalTok{x }\OtherTok{=} \FunctionTok{fisher\_support}\NormalTok{(}\AttributeTok{p =}\NormalTok{ p, }\AttributeTok{n1 =}\NormalTok{ n1, }\AttributeTok{n2 =}\NormalTok{ n2)}
\FunctionTok{plot}\NormalTok{(x, }\FunctionTok{pdf\_Fisher}\NormalTok{(x, }\AttributeTok{p =}\NormalTok{ p, }\AttributeTok{n1 =}\NormalTok{ n1, }\AttributeTok{n2 =}\NormalTok{ n2), }\AttributeTok{type=}\StringTok{\textquotesingle{}l\textquotesingle{}}\NormalTok{, }\AttributeTok{lwd=}\DecValTok{2}\NormalTok{, }\AttributeTok{col=}\StringTok{"red"}\NormalTok{)}
\NormalTok{p }\OtherTok{=} \DecValTok{75}
\NormalTok{n1 }\OtherTok{=} \DecValTok{100}
\NormalTok{n2 }\OtherTok{=} \DecValTok{200}
\NormalTok{x }\OtherTok{=} \FunctionTok{fisher\_support}\NormalTok{(}\AttributeTok{p =}\NormalTok{ p, }\AttributeTok{n1 =}\NormalTok{ n1, }\AttributeTok{n2 =}\NormalTok{ n2)}
\FunctionTok{lines}\NormalTok{(x, }\FunctionTok{pdf\_Fisher}\NormalTok{(x, }\AttributeTok{p =}\NormalTok{ p, }\AttributeTok{n1 =}\NormalTok{ n1, }\AttributeTok{n2 =}\NormalTok{ n2), }\AttributeTok{type=}\StringTok{\textquotesingle{}l\textquotesingle{}}\NormalTok{, }\AttributeTok{lwd=}\DecValTok{2}\NormalTok{, }\AttributeTok{col=}\StringTok{"darkgreen"}\NormalTok{, }\AttributeTok{ylab=}\StringTok{""}\NormalTok{)}
\NormalTok{p }\OtherTok{=} \DecValTok{25}
\NormalTok{n1 }\OtherTok{=} \DecValTok{100}
\NormalTok{n2 }\OtherTok{=} \DecValTok{200}
\NormalTok{x }\OtherTok{=} \FunctionTok{fisher\_support}\NormalTok{(}\AttributeTok{p =}\NormalTok{ p, }\AttributeTok{n1 =}\NormalTok{ n1, }\AttributeTok{n2 =}\NormalTok{ n2)}
\FunctionTok{lines}\NormalTok{(x, }\FunctionTok{pdf\_Fisher}\NormalTok{(x, }\AttributeTok{p =}\NormalTok{ p, }\AttributeTok{n1 =}\NormalTok{ n1, }\AttributeTok{n2 =}\NormalTok{ n2), }\AttributeTok{type=}\StringTok{\textquotesingle{}l\textquotesingle{}}\NormalTok{, }\AttributeTok{lwd=}\DecValTok{2}\NormalTok{, }\AttributeTok{col=}\StringTok{"darkblue"}\NormalTok{, }\AttributeTok{ylab=}\StringTok{""}\NormalTok{)}
\FunctionTok{legend}\NormalTok{(}\StringTok{"topright"}\NormalTok{, }\FunctionTok{c}\NormalTok{(}\StringTok{"(50,100,100)"}\NormalTok{, }\StringTok{"(75,100,200)"}\NormalTok{,}\StringTok{"(25,100,200)"}\NormalTok{), }\AttributeTok{col=}\FunctionTok{c}\NormalTok{(}\StringTok{"red"}\NormalTok{, }\StringTok{"darkgreen"}\NormalTok{,}\StringTok{"darkblue"}\NormalTok{))}
\end{Highlighting}
\end{Shaded}

\includegraphics{A3_markdown_files/figure-latex/unnamed-chunk-3-1.pdf}

\subsection{Question 2}\label{question-2}

You can also embed plots, for example:

\includegraphics{A3_markdown_files/figure-latex/pressure-1.pdf}

\subsection{Question 3}\label{question-3}

\subsection{Reference}\label{reference}

\end{document}
