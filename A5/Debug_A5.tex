% Options for packages loaded elsewhere
\PassOptionsToPackage{unicode}{hyperref}
\PassOptionsToPackage{hyphens}{url}
%
\documentclass[
]{article}
\usepackage{amsmath,amssymb}
\usepackage{iftex}
\ifPDFTeX
  \usepackage[T1]{fontenc}
  \usepackage[utf8]{inputenc}
  \usepackage{textcomp} % provide euro and other symbols
\else % if luatex or xetex
  \usepackage{unicode-math} % this also loads fontspec
  \defaultfontfeatures{Scale=MatchLowercase}
  \defaultfontfeatures[\rmfamily]{Ligatures=TeX,Scale=1}
\fi
\usepackage{lmodern}
\ifPDFTeX\else
  % xetex/luatex font selection
\fi
% Use upquote if available, for straight quotes in verbatim environments
\IfFileExists{upquote.sty}{\usepackage{upquote}}{}
\IfFileExists{microtype.sty}{% use microtype if available
  \usepackage[]{microtype}
  \UseMicrotypeSet[protrusion]{basicmath} % disable protrusion for tt fonts
}{}
\makeatletter
\@ifundefined{KOMAClassName}{% if non-KOMA class
  \IfFileExists{parskip.sty}{%
    \usepackage{parskip}
  }{% else
    \setlength{\parindent}{0pt}
    \setlength{\parskip}{6pt plus 2pt minus 1pt}}
}{% if KOMA class
  \KOMAoptions{parskip=half}}
\makeatother
\usepackage{xcolor}
\usepackage[margin=1in]{geometry}
\usepackage{color}
\usepackage{fancyvrb}
\newcommand{\VerbBar}{|}
\newcommand{\VERB}{\Verb[commandchars=\\\{\}]}
\DefineVerbatimEnvironment{Highlighting}{Verbatim}{commandchars=\\\{\}}
% Add ',fontsize=\small' for more characters per line
\usepackage{framed}
\definecolor{shadecolor}{RGB}{248,248,248}
\newenvironment{Shaded}{\begin{snugshade}}{\end{snugshade}}
\newcommand{\AlertTok}[1]{\textcolor[rgb]{0.94,0.16,0.16}{#1}}
\newcommand{\AnnotationTok}[1]{\textcolor[rgb]{0.56,0.35,0.01}{\textbf{\textit{#1}}}}
\newcommand{\AttributeTok}[1]{\textcolor[rgb]{0.13,0.29,0.53}{#1}}
\newcommand{\BaseNTok}[1]{\textcolor[rgb]{0.00,0.00,0.81}{#1}}
\newcommand{\BuiltInTok}[1]{#1}
\newcommand{\CharTok}[1]{\textcolor[rgb]{0.31,0.60,0.02}{#1}}
\newcommand{\CommentTok}[1]{\textcolor[rgb]{0.56,0.35,0.01}{\textit{#1}}}
\newcommand{\CommentVarTok}[1]{\textcolor[rgb]{0.56,0.35,0.01}{\textbf{\textit{#1}}}}
\newcommand{\ConstantTok}[1]{\textcolor[rgb]{0.56,0.35,0.01}{#1}}
\newcommand{\ControlFlowTok}[1]{\textcolor[rgb]{0.13,0.29,0.53}{\textbf{#1}}}
\newcommand{\DataTypeTok}[1]{\textcolor[rgb]{0.13,0.29,0.53}{#1}}
\newcommand{\DecValTok}[1]{\textcolor[rgb]{0.00,0.00,0.81}{#1}}
\newcommand{\DocumentationTok}[1]{\textcolor[rgb]{0.56,0.35,0.01}{\textbf{\textit{#1}}}}
\newcommand{\ErrorTok}[1]{\textcolor[rgb]{0.64,0.00,0.00}{\textbf{#1}}}
\newcommand{\ExtensionTok}[1]{#1}
\newcommand{\FloatTok}[1]{\textcolor[rgb]{0.00,0.00,0.81}{#1}}
\newcommand{\FunctionTok}[1]{\textcolor[rgb]{0.13,0.29,0.53}{\textbf{#1}}}
\newcommand{\ImportTok}[1]{#1}
\newcommand{\InformationTok}[1]{\textcolor[rgb]{0.56,0.35,0.01}{\textbf{\textit{#1}}}}
\newcommand{\KeywordTok}[1]{\textcolor[rgb]{0.13,0.29,0.53}{\textbf{#1}}}
\newcommand{\NormalTok}[1]{#1}
\newcommand{\OperatorTok}[1]{\textcolor[rgb]{0.81,0.36,0.00}{\textbf{#1}}}
\newcommand{\OtherTok}[1]{\textcolor[rgb]{0.56,0.35,0.01}{#1}}
\newcommand{\PreprocessorTok}[1]{\textcolor[rgb]{0.56,0.35,0.01}{\textit{#1}}}
\newcommand{\RegionMarkerTok}[1]{#1}
\newcommand{\SpecialCharTok}[1]{\textcolor[rgb]{0.81,0.36,0.00}{\textbf{#1}}}
\newcommand{\SpecialStringTok}[1]{\textcolor[rgb]{0.31,0.60,0.02}{#1}}
\newcommand{\StringTok}[1]{\textcolor[rgb]{0.31,0.60,0.02}{#1}}
\newcommand{\VariableTok}[1]{\textcolor[rgb]{0.00,0.00,0.00}{#1}}
\newcommand{\VerbatimStringTok}[1]{\textcolor[rgb]{0.31,0.60,0.02}{#1}}
\newcommand{\WarningTok}[1]{\textcolor[rgb]{0.56,0.35,0.01}{\textbf{\textit{#1}}}}
\usepackage{graphicx}
\makeatletter
\def\maxwidth{\ifdim\Gin@nat@width>\linewidth\linewidth\else\Gin@nat@width\fi}
\def\maxheight{\ifdim\Gin@nat@height>\textheight\textheight\else\Gin@nat@height\fi}
\makeatother
% Scale images if necessary, so that they will not overflow the page
% margins by default, and it is still possible to overwrite the defaults
% using explicit options in \includegraphics[width, height, ...]{}
\setkeys{Gin}{width=\maxwidth,height=\maxheight,keepaspectratio}
% Set default figure placement to htbp
\makeatletter
\def\fps@figure{htbp}
\makeatother
\setlength{\emergencystretch}{3em} % prevent overfull lines
\providecommand{\tightlist}{%
  \setlength{\itemsep}{0pt}\setlength{\parskip}{0pt}}
\setcounter{secnumdepth}{-\maxdimen} % remove section numbering
\usepackage{fancyhdr}
\usepackage{amsmath}
\pagestyle{fancy}
\fancyhead[CO,CE]{STAT3017/STAT6017 - Big Data Statistics - Sem 2 2023}
\fancyhead[LO,LE]{}
\fancypagestyle{plain}{\pagestyle{fancy}}
\renewcommand{\headrulewidth}{0.4pt}
\ifLuaTeX
  \usepackage{selnolig}  % disable illegal ligatures
\fi
\IfFileExists{bookmark.sty}{\usepackage{bookmark}}{\usepackage{hyperref}}
\IfFileExists{xurl.sty}{\usepackage{xurl}}{} % add URL line breaks if available
\urlstyle{same}
\hypersetup{
  pdftitle={Assignment 5},
  pdfauthor={Songze Yang u7192786},
  hidelinks,
  pdfcreator={LaTeX via pandoc}}

\title{Assignment 5}
\author{Songze Yang u7192786}
\date{}

\begin{document}
\maketitle

\begin{Shaded}
\begin{Highlighting}[]
\NormalTok{knitr}\SpecialCharTok{::}\NormalTok{opts\_chunk}\SpecialCharTok{$}\FunctionTok{set}\NormalTok{(}\AttributeTok{echo =} \ConstantTok{TRUE}\NormalTok{)}
\NormalTok{knitr}\SpecialCharTok{::}\NormalTok{opts\_chunk}\SpecialCharTok{$}\FunctionTok{set}\NormalTok{(}\AttributeTok{eval=}\ConstantTok{FALSE}\NormalTok{)}
\FunctionTok{library}\NormalTok{(mvnfast)}
\FunctionTok{library}\NormalTok{(RMTstat)}
\FunctionTok{library}\NormalTok{(future.apply)}
\end{Highlighting}
\end{Shaded}

\begin{verbatim}
## Loading required package: future
\end{verbatim}

\begin{Shaded}
\begin{Highlighting}[]
\FunctionTok{library}\NormalTok{(knitr)}
\FunctionTok{library}\NormalTok{(expm)}
\end{Highlighting}
\end{Shaded}

\begin{verbatim}
## Loading required package: Matrix
\end{verbatim}

\begin{verbatim}
## 
## Attaching package: 'expm'
\end{verbatim}

\begin{verbatim}
## The following object is masked from 'package:Matrix':
## 
##     expm
\end{verbatim}

Question 2

In this question, we shall consider high-dimensional sample covariance
matrices of data that is sampled from an elliptical distribution. We say
that a random vector \(\mathbf{x}\) with zero mean follows an elliptical
distribution if (and only if) it has the stochastic representation
\[ \mathbf{x} = \xi A \mathbf{u}, \quad (\star) \]where the matrix
\(A \in \mathbb{R}^{p \times p}\) is nonrandom and
\(\text{rank}(A) = p\), \(\xi \geq 0\) is a random variable representing
the radius of \(\mathbf{x}\), and \(u \in \mathbb{R}^p\) is the random
direction, which is independent of \(\xi\) and uniformly distributed on
the unit sphere \(S_{p-1}\) in \(\mathbb{R}^p\), denoted by
\(\mathbf{u} \sim \text{Unif}(S_{p-1})\). The class of elliptical
distributions is a natural generalization of the multivariate normal
distribution, and contains many widely used distributions as special
cases including the multivariate t-distribution, the symmetric
multivariate Laplace distribution, and the symmetric multivariate stable
distribution.

\begin{enumerate}
\def\labelenumi{(\alph{enumi})}
\tightlist
\item
  Write a function runifsphere(n,p) that samples \(n\) observations from
  the distribution \(\text{Unif}(S_{p-1})\) using the fact that if
  \(\mathbf{z} \sim N_p(0, I_p)\) then
  \(\frac{\mathbf{z}}{\| \mathbf{z} \|} \sim \text{Unif}(S_{p-1})\).
  Check your results by:
\end{enumerate}

set \(p = 25\), \(n = 50\) and show that the (Euclidean) norm of each
observation is equal to 1.

Answer to question 2 (a) (1):

Now we show that the (Euclidean) norm of each observation is equal to 1
in (1):

generate a scatter plot in the case \(p = 2\), \(n = 500\) to show that
the samples lie on a circle.

Answer to question 2 (a) (2):

Now we show that the observations lie on a circle in (2):

generate a scatter plot in the case \(p = 2\), \(n = 500\) to show that
the samples lie on a circle.

Answer to question 2 (a) (2):

Now we show that the observations lie on a circle in (2):

We can see that the plot matches the distribution very well.

Suppose that \(\mathbf{x}_1, \mathbf{x}_2, \dots, \mathbf{x}_n\) are
\(p\)-dimensional observations sampled from an elliptic distribution
\((\star)\). We stack these observations into the data matrix
\(\mathbf{X}\) and calculate the sample covariance matrix
\(\mathbf{S}_n := \mathbf{X} \mathbf{X}^T / n\). Theorem 2.2 of the
recent paper {[}C{]} is a central limit theorem for linear spectral
statistics (LSS) of \(\mathbf{S}_n\). For example, Eq. (2.10) in {[}C{]}
provides the case of the joint distribution of the LSS \(\phi_1(x) = x\)
and \(\phi_2(x) = x^2\). Following the notation used there (for all the
following terms in this question). Perform a simulation experiment to
examine the fluctuations of \(\hat{\beta}_{n1}\) and
\(\hat{\beta}_{n2}\). In the experiment, take
\(H_p = \frac{1}{2} \delta_1 + \frac{1}{2} \delta_2\) and choose the
distribution of \(\xi \sim k_1 \text{Gamma}(p, 1)\) with
\(k_1 = \frac{1}{\sqrt{p + 1}}\). Set the dimensions to be \(p = 200\)
and \(n = 400\). Choose the number of simulations based on the
computational power of your machine. Similar to Figure 1 in {[}C{]}, use
a QQ-plot to show normality.

Answer to question 2 (b):

The population \(PSD \ H_p\) is assumed to be fixed and therefore we
have \(H_p = H\). Immediately, we have the following conclusion:

\[
\gamma_{nj} = \int_{} t^jdH_p(t) = \int_{} t^jdH(t) = \gamma_j
\]

In this question, we assume that the
\(H_p = \frac{1}{2} \delta_1 + \frac{1}{2} \delta_2 \Rightarrow \Sigma = diag(1,..,1,2,...,2)=AA^T\)
with equal number of 1's and 2's so we can compute their respectively
values:

\begin{align*}
\gamma_1 = \gamma_{n1} = \int t dH_p(t) = \int t d(\frac{1}{2}\delta_1 + \frac{1}{2}\delta_2) = \frac{1}{2} \int t d\delta_1(t) + \frac{1}{2} \int t d \delta_2(t) \\
\gamma_2 = \gamma_{n2} = \int t^2 dH_p(t) = \int t^2 d(\frac{1}{2}\delta_1 + \frac{1}{2}\delta_2) = \frac{1}{2} \int t^2 d\delta_1(t) + \frac{1}{2} \int t^2 d \delta_2(t)
\end{align*}

By property:

\[
\begin{equation*}
\int f(t) \delta_a(t)dt = f(a) \tag{1}
\end{equation*}
\]

We have that: \begin{align*}
\int t d \delta_2(t) = 1\\
\int t d \delta_2(t) = 2\\
\int t^2 d\delta_1(t) = 1^2\\
\int t^2 d\delta_2(t) = 2^2\\
\end{align*}

\[
\gamma_1  = \gamma_{n1} = 1.5 \quad \text{and} \quad \gamma_2 = \gamma_{n2} =2.5 \quad \text{similarly} \quad \gamma_3 = 4.5 \quad \text{and} \quad \gamma_4 = 8.5
\]

The value we need to show the CLT assuming
\(p = 200, \quad n = 400 \quad \text{and} \quad \frac{p}{n} :=c_n = c = \frac{1}{2}\):

\begin{align*}
\tau & = 4 \\
\beta_{n1} & = \gamma_{n1} = 1.5, \\
\beta_{n2} & = \gamma_{n2} + c_n\gamma^2_{n1} = 2.5 + \frac{1}{2} \times 1.5^2 = 3.625, \\
\nu_1 & = 0, \\
\nu_2 & = c\gamma_2 + c(\tau - 2)\gamma_1, \\
\psi_{11} & = 2c\gamma_2 + c(\tau - 2)\gamma^2_1 = 2.5 + 1.5^2 = 4.75, \\
\psi_{22} & = 8c\gamma_4 + 4c^2\gamma^2_2 + 16c^2\gamma_1\gamma_3 + 8c^3\gamma^2_1 \gamma_2 \\
& \quad + 4c(\tau - 2)(c\gamma^2_1 + \gamma_2)^2 = 125.4375
\end{align*}

Let's code up these variables into R:

\end{document}
